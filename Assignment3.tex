% !TEX TS-program = pdflatex
% !TEX encoding = UTF-8 Unicode

% This is a simple template for a LaTeX document using the "article" class.
% See "book", "report", "letter" for other types of document.

\documentclass[11pt]{article} % use larger type; default would be 10pt

\usepackage[utf8]{inputenc} % set input encoding (not needed with XeLaTeX)

%%% Examples of Article customizations
% These packages are optional, depending whether you want the features they provide.
% See the LaTeX Companion or other references for full information.

%%% PAGE DIMENSIONS
\usepackage{geometry} % to change the page dimensions
\geometry{a4paper} % or letterpaper (US) or a5paper or....
% \geometry{margin=2in} % for example, change the margins to 2 inches all round
% \geometry{landscape} % set up the page for landscape
%   read geometry.pdf for detailed page layout information

\usepackage{graphicx} % support the \includegraphics command and options

% \usepackage[parfill]{parskip} % Activate to begin paragraphs with an empty line rather than an indent

%%% PACKAGES
\usepackage{booktabs} % for much better looking tables
\usepackage{array} % for better arrays (eg matrices) in maths
\usepackage{paralist} % very flexible & customisable lists (eg. enumerate/itemize, etc.)
\usepackage{verbatim} % adds environment for commenting out blocks of text & for better verbatim
\usepackage{subfig} % make it possible to include more than one captioned figure/table in a single float
\usepackage{tabto}
% These packages are all incorporated in the memoir class to one degree or another...

%%% HEADERS & FOOTERS
\usepackage{fancyhdr} % This should be set AFTER setting up the page geometry
\pagestyle{fancy} % options: empty , plain , fancy
\renewcommand{\headrulewidth}{0pt} % customise the layout...
\lhead{}\chead{}\rhead{}
\lfoot{}\cfoot{\thepage}\rfoot{}

%%% SECTION TITLE APPEARANCE
\usepackage{sectsty}
\allsectionsfont{\sffamily\mdseries\upshape} % (See the fntguide.pdf for font help)
% (This matches ConTeXt defaults)

%%% ToC (table of contents) APPEARANCE
\usepackage[nottoc,notlof,notlot]{tocbibind} % Put the bibliography in the ToC
\usepackage[titles,subfigure]{tocloft} % Alter the style of the Table of Contents
\renewcommand{\cftsecfont}{\rmfamily\mdseries\upshape}
\renewcommand{\cftsecpagefont}{\rmfamily\mdseries\upshape} % No bold!

%%% END Article customizations

%%% The "real" document content comes below...

\title{Assignment 3 Hash/Heap }
\author{Dylan Taylor}
%\date{} % Activate to display a given date or no date (if empty),
         % otherwise the current date is printed 

\begin{document}

\maketitle 
	
\begin{enumerate}
\item What occupancy ratio should you use? \\\\
My hash function was as follows:\\\\
       	int hashed = 0;\\
        int multiplier = 1;\\
        for(int i = 0; i < word.length(); i++)\{\\
            hashed += (int)word[i] * multiplier;\\
            multiplier +=13;\\
        \}\\
        hashed = abs(hashed);\\
        hashed = hashed \% SIZE;\\\\
I tested the following: Array size, collision rate, time to create the hash table, and finally the time it took to sort the hash table with heap sort.\\

Array size: 100000000 	cr:15\% time 43 milli	sort: 6 milli\\
Array size: 100 	  	cr:98\% time 63 milli	sort: 3 milli\\
Array size: 1000		cr:88\% time 45 milli	sort: 2 milli\\
Array size: 10000		cr:41\% time 38 milli	sort: 2 milli\\
Array size: 100000		cr:16\% time 38 milli	sort: 2 milli \\\\
My findings show that an array size of 100,000 with my hash function seemed to be the best for collision and for run time.\\\\
\item  How do you set up a hash table of variable size?	\\

You would either have to create a linked list which in reality would no longer be a hash table, or you would have to create a new array of a larger size and then copy all the old values over and continue.\\

\item What hash function? \\

       	int hashed = 0;\\
        int multiplier = 1;\\
        for(int i = 0; i < word.length(); i++)\{\\
            hashed += (int)word[i] * multiplier;\\
            multiplier +=13;\\
        \}\\
        hashed = abs(hashed);\\
        hashed = hashed \% SIZE;\\\\
\item  What if there are collisions in the table?\\

I used chaining so I made my hash table an array of linked lists. I found this easier to keep track of and to handle infinite collisions. Double hashing with large amounts of data can lead to errors.\\

\item  Do you need an interface file (a “.h” file) for these functions? \\

You do not "need" an interface file, however I used multiple as I found it easier and clearer to read and edit my program.\\

\end{enumerate}




\end{document}
